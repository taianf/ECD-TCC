\section{\textit{Gradient boosting} em árvore}
\label{sec:gbt}

Para treinar o modelo de \textit{gradient boosting} em árvore, a matriz de parâmetros na tabela~\ref{tab:param-gbt} foi utilizada.
Pela equação~\ref{eq:qtd-modelos}, foram treinados 27 ($3 \times 3 \times 3$) modelos.
Os parâmetros selecionados estão presentes na tabela~\ref{tab:param-final-gbt}.

\begin{table}[h]
    \centering
    \begin{tabular}{|c|c|c|}
        \hline
        Parâmetro         & Valores    \\ \hline
        \textit{maxIter}  & 10, 20, 30 \\
        \textit{maxDepth} & 2, 6, 10   \\ \hline
    \end{tabular}
    \caption{Parâmetros para treino do \textit{gradient boosting} em árvore}
    \label{tab:param-gbt}
\end{table}

\begin{table}[h]
    \centering
    \begin{tabular}{|c|c|c|}
        \hline
        Parâmetro         & Valores \\ \hline
        \textit{maxIter}  & 20      \\
        \textit{maxDepth} & 6       \\ \hline
    \end{tabular}
    \caption{Parâmetros selecionados na validação cruzada do \textit{gradient boosting} em árvore}
    \label{tab:param-final-gbt}
\end{table}

\textit{maxIter}: a quantidade máxima de iterações do algoritmo.

\textit{maxDepth}: a profundidade máxima das árvores geradas.
