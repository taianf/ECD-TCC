\section{Floresta Aleatória}
\label{sec:rf}

Para treinar a floresta aleatória, a matriz de parâmetros na tabela~\ref{tab:param-rf} foi utilizada.
Pela equação~\ref{eq:qtd-modelos}, foram treinados 45 ($3 \times 3 \times 5$) modelos.
Os parâmetros selecionados estão presentes na tabela~\ref{tab:param-final-rf}.

\begin{table}[h]
    \centering
    \begin{tabular}{|c|c|c|}
        \hline
        Parâmetro         & Valores           \\ \hline
        \textit{numTrees} & 10, 20, 30        \\
        \textit{maxDepth} & 5, 10, 15, 25, 30 \\ \hline
    \end{tabular}
    \caption{Parâmetros para treino da floresta aleatória}
    \label{tab:param-rf}
\end{table}

\begin{table}[h]
    \centering
    \begin{tabular}{|c|c|c|}
        \hline
        Parâmetro         & Valores \\ \hline
        \textit{numTrees} & 20      \\
        \textit{maxDepth} & 15      \\ \hline
    \end{tabular}
    \caption{Parâmetros selecionados na validação cruzada da floresta aleatória}
    \label{tab:param-final-rf}
\end{table}

\textit{numTrees}: o número de árvores que devem compor a floresta aleatória.

\textit{maxDepth}: a profundidade máxima das árvores geradas.
