\section{Máquina de vetores de suporte}
\label{sec:lsvc}

Para treinar o modelo de Máquina de vetores de suporte, a matriz de parâmetros na tabela~\ref{tab:param-lsvc} foi utilizada.
Pela equação~\ref{eq:qtd-modelos}, foram treinados 27 ($3 \times 3 \times 3$) modelos.
Os parâmetros selecionados estão presentes na tabela~\ref{tab:param-final-lsvc}.

\begin{table}[h]
    \centering
    \begin{tabular}{|c|c|c|}
        \hline
        Parâmetro         & Valores         \\ \hline
        \textit{maxIter}  & 1, 2, 3         \\
        \textit{regParam} & 0.01, 0.02, 0.1 \\ \hline
    \end{tabular}
    \caption{Parâmetros para treino da Máquina de vetores de suporte}
    \label{tab:param-lsvc}
\end{table}

\begin{table}[h]
    \centering
    \begin{tabular}{|c|c|c|}
        \hline
        Parâmetro         & Valores \\ \hline
        \textit{maxIter}  & 3       \\
        \textit{regParam} & 0.02    \\ \hline
    \end{tabular}
    \caption{Parâmetros selecionados na validação cruzada da Máquina de vetores de suporte}
    \label{tab:param-final-lsvc}
\end{table}

\textit{maxIter}: a quantidade máxima de iterações do algoritmo.

\textit{regParam}: é o parâmetro de regularização.
A regularização serve para prevenir \textit{overfitting}, penalizando modelos com valores de coeficiente muito altos.
Coeficientes muito altos podem ocorrer quando há algum parâmetro raro que força uma correlação muito alta com a variável a ser estimada.
