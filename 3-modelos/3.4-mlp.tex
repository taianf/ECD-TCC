\section{Perceptron multicamadas}
\label{sec:mlp}

Para treinar o modelo de perceptron multicamadas, a matriz de parâmetros na tabela~\ref{tab:param-mlp} foi utilizada.
Pela equação~\ref{eq:qtd-modelos}, foram treinados 18 ($3 \times 6$) modelos.
Os parâmetros selecionados estão presentes na tabela~\ref{tab:param-final-mlp}.

\begin{table}[h]
    \centering
    \begin{tabular}{|c|c|c|}
        \hline
        Parâmetro       & Valores                  \\ \hline
        \textit{layers} & \makecell{(12, 5, 5, 2),\\(12, 10, 10, 2),\\(12, 20, 20, 2),\\(12, 5, 5, 5, 2),\\(12, 10, 10, 10, 2),\\(12, 20, 20, 20, 2)} \\ \hline
    \end{tabular}
    \caption{Parâmetros para treino do perceptron multicamadas}
    \label{tab:param-mlp}
\end{table}

\begin{table}[h]
    \centering
    \begin{tabular}{|c|c|c|}
        \hline
        Parâmetro       & Valores             \\ \hline
        \textit{layers} & (12, 10, 10, 10, 2) \\ \hline
    \end{tabular}
    \caption{Parâmetros selecionados na validação cruzada do perceptron multicamadas}
    \label{tab:param-final-mlp}
\end{table}

\textit{layers}: uma sequência com a quantidade de neurônios em cada camada.
