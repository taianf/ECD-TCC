\section{Máquinas de fatoração}
\label{sec:fm}

Para treinar o modelo de Máquinas de fatoração a matriz de parâmetros na tabela~\ref{tab:param-fm} foi utilizada.
Pela equação~\ref{eq:qtd-modelos}, foram treinados 135 ($3 \times 3 \times 3 \times 5$) modelos.
Os parâmetros selecionados estão presentes na tabela~\ref{tab:param-final-fm}.

\begin{table}[h]
    \centering
    \begin{tabular}{|c|c|c|}
        \hline
        Parâmetro         & Valores                       \\ \hline
        \textit{maxIter}  & 1, 5, 10                      \\
        \textit{regParam} & 0.01, 0.02, 0.1               \\
        \textit{stepSize} & 0.001, 0.005, 0.01, 0.05, 0.1 \\ \hline
    \end{tabular}
    \caption{Parâmetros para treino das Máquinas de fatoração}
    \label{tab:param-fm}
\end{table}

\begin{table}[h]
    \centering
    \begin{tabular}{|c|c|c|}
        \hline
        Parâmetro         & Valores \\ \hline
        \textit{maxIter}  & 10      \\
        \textit{regParam} & 0.02    \\
        \textit{stepSize} & 0.1     \\ \hline
    \end{tabular}
    \caption{Parâmetros selecionados na validação cruzada das Máquinas de fatoração}
    \label{tab:param-final-fm}
\end{table}

\textit{maxIter}: a quantidade máxima de iterações do algoritmo.

\textit{regParam}: é o parâmetro de regularização.
A regularização serve para prevenir \textit{overfitting}, penalizando modelos com valores de coeficiente muito altos.
Coeficientes muito altos podem ocorrer quando há algum parâmetro raro que força uma correlação muito alta com a variável a ser estimada.

\textit{stepSize}: o tamanho do incremento a cada iteração.
