\chapter{Metodologia}
\label{ch:metodologia}

Para alcançar o objetivo deste trabalho, que consiste em analisar o risco de morte em pacientes hospitalizados com COVID-19, conforme as comorbidades, utilizou-se alguns algoritmos de aprendizado de máquina para tentar predizer se o paciente com COVID-19 sobreviverá ou não a esta doença.
Os dados dos hospitalizados foram obtidos no portal OPENDATASUS~\cite{opendatasus2020,opendatasus20212022} no dia 12 de outubro.

As comorbidades catalogadas pelo OPENDATASUS e utilizadas neste trabalho foram:

\begin{itemize}
    \item Doença Cardiovascular Crônica
    \item Doença Hematológica Crônica
    \item Síndrome de Down
    \item Doença Hepática Crônica
    \item Asma
    \item Diabetes mellitus
    \item Doença Neurológica Crônica
    \item Outra Pneumatopatia Crônica
    \item Imunodeficiência ou Imunodepressão
    \item Doença Renal Crônica
    \item Obesidade
\end{itemize}

Os dados disponibilizados foram filtrados para apenas os casos confirmados como COVID-19 e com evolução desconhecida foram descartados.
Como alguns algoritmos são especialmente sensíveis quando a classe de avaliação é desbalanceada, e como a base possui mais curados do que óbitos, foi realizado um \textit{undersampling}, foram sorteados aleatoriamente uma quantidade de curados dentre a base de entrada para que os dados de treino tenha uma quantidade semelhante de dados entre as classes.

Os algoritmos foram treinados com Apache Spark, ``um motor multilinguagem para executar engenharia de dados, ciências de dados e aprendizado de máquina em máquinas de nó único ou `clusters'\,''~\cite{apachespark}.
O Spark permite realizar validação cruzada para treinar o modelo e utilizar uma matriz de parâmetros para selecionar automaticamente o melhor conjunto de parâmetros, através do avaliador selecionado~\cite{apachesparkcv}.
Dentre os algoritmos implementados no Spark~\cite{apachesparkml}, este trabalho testou os seguintes:

\begin{itemize}
    \item Regressão logística
    \item Floresta aleatória
    \item \textit{Gradient boosting} em árvore
    \item Perceptron multicamadas
    \item Máquina de vetores de suporte
    \item \textit{Naïve Bayes}
    \item Máquinas de Fatoração
\end{itemize}

A regressão logística calcula valores que possuem interpretação matemática para modelar a probabilidade de um evento ocorrer, conforme a presença de outro.
Por isso, esses valores serão utilizados para ranquear as comorbidades com maior influência no risco de morte.
As matrizes de parâmetros serão abordadas em~\ref{ch:modelos}.
O código utilizado nesse trabalho, feito na linguagem Scala, está disponível no repositório do GitHub \href{https://github.com/taianf/ECD-TCC-Dados}~\cite{githubtcc}.
