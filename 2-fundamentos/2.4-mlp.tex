\section{Perceptron multicamadas}
\label{sec:fun:mlp}

Um perceptron multicamadas (Multilayer perceptron ou MLP em inglês) é uma classe de rede neural artificial feedforward totalmente ligada.
O termo MLP é utilizado ambiguamente, por vezes de forma solta para significar qualquer rede neural artificial de alimentação, por vezes estritamente para se referir a redes compostas de múltiplas camadas de perceptrons.
Os perceptrons multicamadas são por vezes referidos coloquialmente como redes neurais ``baunilha'', especialmente quando têm uma única camada oculta~\cite{elementsofstatisticallearning}.

Um MLP consiste em pelo menos três camadas de nós: uma camada de entrada, uma camada oculta e uma camada de saída.
Com excepção dos nós de entrada, cada nó é um neurônio que utiliza uma função de ativação não linear.
O MLP utiliza uma técnica de aprendizado supervisionado chamada \textit{backpropagation} para treino.
As suas múltiplas camadas e ativação não linear distinguem o MLP de um perceptron linear e pode distinguir os dados que não são separáveis linearmente~\cite{perceptrons}.
