\chapter{Fundamentos}
\label{ch:fundamentos}

O aprendizado de máquinas (Machine Learning, em inglês) é um campo de estudo dedicado à compreensão e construção de métodos que `aprendem', ou seja, métodos que usam os dados para melhorar o desempenho em algum conjunto de tarefas~\cite{mitcheltom}.
Os algoritmos de aprendizado de máquinas constroem um modelo baseado em dados de amostra, conhecidos como dados de treino, a fim de fazer previsões ou tomar decisões sem serem explicitamente programados para isso~\cite{simonphill}.

Os métodos de aprendizado de máquina são classificados basicamente em três tipos~\cite{russellstuart}:

\begin{itemize}
    \item Aprendizado supervisionado:
    \subitem Exemplos de entradas e saídas pré-classificadas são apresentados ao computador.
    O objetivo é aprender uma regra geral que mapeia entradas para saídas.

    \item Aprendizado não supervisionado:
    \subitem Nenhum tipo de classificação é dado ao algoritmo de aprendizado, deixando-o sozinho para encontrar relacionamentos nas entradas fornecidas.
    O aprendizado não supervisionado pode ser um objetivo em si, descobrir novos padrões nos dados, ou um meio para um fim.

    \item Aprendizado por reforço:
    \subitem Um programa de computador interage com um ambiente dinâmico, no qual o programa deve realizar um determinado objetivo (por exemplo, dirigir um veículo).
    O feedback sobre recompensas e punições é fornecido ao programa à medida que ele navega no espaço do problema.
    Outro exemplo de aprendizado por reforço é aprender a jogar um determinado jogo apenas jogando contra um oponente.
\end{itemize}

Existem algumas técnicas para juntar vários modelos em um só, como \textit{ensemble} e \textit{boosting}.

Os métodos \textit{ensemble} utilizam múltiplos algoritmos de aprendizado para obter um melhor desempenho preditivo do que o que poderia ser obtido apenas com qualquer um dos algoritmos de aprendizado constituintes.
Ao contrário de um \textit{ensemble} estatístico em mecânica estatística, que é geralmente infinito, um \textit{ensemble} de aprendizado de máquinas consiste apenas num conjunto finito concreto de modelos alternativos, mas tipicamente permite que exista uma estrutura muito mais flexível entre essas alternativas~\cite{ensemble}.

\textit{Boosting} é um método de \textit{ensemble} que combina um conjunto de modelos fracos num modelo forte para minimizar erros de treino.
No \textit{boosting}, uma amostra aleatória de dados é selecionada, treinada com um modelo e depois treinada sequencialmente, isto é, cada modelo tenta compensar as fraquezas do seu predecessor.
Com cada iteração as regras fracas de cada classificador individual são combinadas para formar uma regra de previsão forte~\cite{boosting}.

Os dados deste trabalho já estão pré-classificados, portanto foram utilizados alguns modelos de aprendizado supervisionado.
Os modelos neste trabalho e citados no capítulo~\ref{ch:metodologia} foram escolhidos pois já tem algoritmos de treinamento presentes no Apache Spark.
Para mais detalhes sobre a implementação dos modelos no Apache Spark é possível conferir a documentação oficial~\cite{apachesparkml}.

\section{Regressão Logística}
\label{sec:fun:lr}

A regressão logística é uma técnica estatística que visa produzir, a partir de um conjunto de observações, um modelo que permita a previsão de valores tomados por uma variável categórica, muitas vezes binária, a partir de uma série de variáveis explicativas contínuas ou binárias.
É um caso especial de modelos Lineares Generalizados que prevê a probabilidade dos resultados~\cite{glmarticle}.

A regressão logística tem a vantagem de ter uma interpretação prática para os coeficientes, onde eles representam a probabilidade de um evento ocorrer em função de outros fatores.
Essa interpretação pode ajudar a descobrir quais variáveis são relevantes para o modelo e quais não são.
Por ser um modelo simples de treinar e possuir coeficientes interpretáveis, é frequentemente utilizado como base para comparação da eficiência de outros modelos treinados.

A interpretação dos coeficientes se dá pela equação~\ref{eq:prob-rl}.
Coeficientes $\beta$ positivos indicam que a probabilidade da classe avaliada aumenta quando o fator é positivo enquanto coeficientes $\beta$ negativos indicam que a probabilidade diminui.

\begin{equation}
    \label{eq:prob-rl}
    p_i = e^{\beta_i}
\end{equation}

Por exemplo, um coeficiente $\beta_i = 0,1$ gera um $p_i = 1,105171$, significando que quando o fator $i$ está presente, a probabilidade da classe alvo ser positiva é aproximadamente 10,5171\% maior se comparado a quando o fator $i$ está ausente.

O intercepto $\beta_0$ representa a probabilidade da classe alvo ser positiva quando todos os coeficientes são 0 e seu cálculo se dá pela equação~\ref{eq:int-rl}.

\begin{equation}
    \label{eq:int-rl}
    p = \frac{e^{\beta_0}}{(1 + e^{\beta_0}) }
\end{equation}

\section{Floresta Aleatória}
\label{sec:fun:rf}

Florestas aleatórias ou florestas de decisão aleatória é um método de aprendizado \textit{ensemble} para classificação, regressão e outras tarefas que opera construindo várias árvores de decisão em tempo de treinamento~\cite{hotimkam}.
O aprendizado de árvore de decisão é uma abordagem de aprendizado supervisionado usada em estatística, mineração de dados e aprendizado de máquina~\cite{decisiontree}.
Uma árvore é construída dividindo o conjunto fonte, que constitui o nó raiz da árvore, em subconjuntos – que constituem os filhos sucessores.
Esse processo é repetido em cada subconjunto.
A recursão é concluída quando o subconjunto em um nó tem todos os mesmos valores que a variável de destino, ou quando a divisão não agrega mais valor às previsões.

Para tarefas de classificação, a saída da floresta aleatória é a classe selecionada pela maioria das árvores.
Para tarefas de regressão, a previsão média das árvores individuais é retornada.
Florestas de decisão aleatórias diminuem a chance de \textit{overfitting} das árvores de decisão ao seu conjunto de treinamento.
As florestas aleatórias geralmente superam as árvores de decisão, mas sua precisão é menor do que o \textit{Gradient boosting} em árvore, visto na seção~\ref{sec:fun:gbt}.

\section{\textit{Gradient boosting} em árvore}
\label{sec:fun:gbt}

O \textit{Gradient boosting} é uma técnica de aprendizado de máquina usada em tarefas de regressão e classificação, entre outras.
Fornece um modelo de previsão sob a forma de um conjunto de modelos de previsão fracos, que são tipicamente árvores de decisão~\cite{gbta1,elementsofstatisticallearning}.
Quando uma árvore de decisão é o modelo fraco, o algoritmo resultante é chamado de árvores de gradiente e normalmente tem um desempenho superior ao da floresta aleatória.


\section{Perceptron multicamadas}
\label{sec:fun:mlp}

Um perceptron multicamadas (Multilayer perceptron ou MLP em inglês) é uma classe de rede neural artificial feedforward totalmente ligada.
O termo MLP é utilizado ambiguamente, por vezes de forma solta para significar qualquer rede neural artificial de alimentação, por vezes estritamente para se referir a redes compostas de múltiplas camadas de perceptrons.
Os perceptrons multicamadas são por vezes referidos coloquialmente como redes neurais ``baunilha'', especialmente quando têm uma única camada oculta~\cite{elementsofstatisticallearning}.

Um MLP consiste em pelo menos três camadas de nós: uma camada de entrada, uma camada oculta e uma camada de saída.
Com excepção dos nós de entrada, cada nó é um neurônio que utiliza uma função de ativação não linear.
O MLP utiliza uma técnica de aprendizado supervisionado chamada \textit{backpropagation} para treino.
As suas múltiplas camadas e ativação não linear distinguem o MLP de um perceptron linear e pode distinguir os dados que não são separáveis linearmente~\cite{perceptrons}.

\section{Máquina de vetores de suporte}
\label{sec:fun:lsvc}

No aprendizado de máquinas, as máquinas de vetores de suporte (Support Vector Machines, SVMs em inglês) são modelos de aprendizado supervisionados com algoritmos de aprendizado associados que analisam dados para classificação e regressão.
Dado um conjunto de exemplos de formação, cada um marcado como pertencendo a uma de duas categorias, um algoritmo de SVM constrói um modelo que atribui novos exemplos a uma ou outra categoria, tornando-o um classificador linear binário não probabilístico, embora existam métodos para utilizar SVM num cenário de classificação probabilístico.

A SVM mapeia exemplos de formação para pontos no espaço de modo a maximizar a largura do intervalo entre as duas categorias.
Novos exemplos são então mapeados para esse mesmo espaço e prevê-se que pertençam a uma categoria com base em que lado do intervalo caem.
Além de realizar a classificação linear, as SVMs podem realizar eficientemente uma classificação não linear usando o que é chamado de \textit{kernel trick}, mapeando implicitamente suas entradas em espaços de características de dimensões mais altas~\cite{svms}.

\section{Classificadores \textit{Naïve Bayes}}
\label{sec:fun:nb}

Em estatística, os classificadores \textit{Naïve Bayes} são uma família de simples ``classificadores probabilísticos'' baseados na aplicação do teorema Bayes com fortes suposições de independência entre as características.
Eles estão entre os modelos de rede Bayes mais simples, mas podem atingir altos níveis de precisão~\cite{bayes,elementsofstatisticallearning}.

Os classificadores Naive Bayes são altamente escaláveis, exigindo uma série de parâmetros lineares no número de variáveis (características/previsões) em um problema de aprendizado.
O treinamento de máxima probabilidade pode ser feito avaliando uma expressão de forma fechada, que leva tempo linear, ao invés de uma aproximação iterativa cara como a utilizada para muitos outros tipos de classificadores.


\section{Máquinas de fatoração}
\label{sec:fun:fm}

As máquinas de fatoração (Factorization Machines - FM em inglês) são capazes de estimar as interações entre as características, mesmo em problemas com grande esparsidade (muitos valores zero ou ausentes).
FM pode ser usada para regressão e o critério de otimização é o erro quadrático médio.
FM também pode ser usada para classificação binária através da função sigmóide e o critério de otimização é a perda logística~\cite{facmachineo,apachesparkml}.

