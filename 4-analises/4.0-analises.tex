\chapter{Análises}
\label{ch:analises}

Os dados disponíveis no OPENDATASUS possuem 3.443.459 entradas e totalizam 1,87GB em arquivos no formato csv (\textit{comma separated values}).
Filtrando os dados de interesse totalizam 990.466 entradas.
Estas entradas foram separadas em 8o\% para treino e validação e 20\% para teste.
Estes 80\% foram usados na validação cruzada.
A cada iteração dois terços desses 80\% foram usados para treino e um terço usado para avaliação dos modelos gerados.
O melhor modelo gerado nas iterações é avaliado com os dados separados para teste.

Para analisar os modelos de classificação gerados podemos usar algumas métricas: \textit{Accuracy}, \textit{Precision}, \textit{Recall} e \textit{F1 Score}.
Estas métricas são calculadas com base nos valores de Verdadeiros Positivos (VP), Falsos Positivos (FP), Falsos Negativos (FN) e Verdadeiros Negativos (VN).
Esse valores aparecem na Matriz de Confusão dos modelos gerados pelo Apache Spark.

\begin{table}[h]
    \centering
    \begin{tabular}{|c|c|c|}
        \hline
        Valor real \textbackslash{} Classificação do modelo & Positivo & Negativo \\ \hline
        Positivo                                            & VP       & FN       \\ \hline
        Negativo                                            & FP       & VN       \\ \hline
    \end{tabular}
    \caption{Matriz de confusão}
    \label{tab:matriz-confusao}
\end{table}

\textit{Accuracy}~(\ref{eq:accuracy}), acurácia em português, é a divisão entre todos os acertos pelo total de previsões.

\begin{equation}
    \label{eq:accuracy}
    \frac{VP + VN}{VP + VN + FP + FN}
\end{equation}

\textit{Precision}~(\ref{eq:precision}), precisão em português, é a divisão entre os Verdadeiros Positivos e todas as previsões positivas.

\begin{equation}
    \label{eq:precision}
    \frac{VP}{VP + FP}
\end{equation}

\textit{Recall}~(\ref{eq:recall}), revocação em português, é a divisão entre os Verdadeiros Positivos e todas os valores que realmente são positivos.

\begin{equation}
    \label{eq:recall}
    \frac{VP}{VP + FN}
\end{equation}

\textit{F1 Score}~(\ref{eq:f1score}), é a média harmônica entre \textit{Precision} e \textit{Recall}.

\begin{equation}
    \label{eq:f1score}
    \frac{2 * Precision * Recall}{Precision + Recall}
\end{equation}

\section{Comparativo entre modelos}
\label{sec:comparativo-entre-modelos}

Uma primeira análise a se fazer, é comparar o custo de treino de cada modelo.
Todos os modelos foram treinados na mesma máquina, cujas especificações encontram-se no apêndice~\ref{ch:maquina}.

\begin{table}[h]
    \centering
    \begin{tabular}{|l|r|r|r|}
        \hline
        Modelo                               & Tempo (s) & Quantidade & Tempo/modelo \\
        \hline
        Regressão Logística                  & 1140      & 135        & 8.4          \\
        Máquinas de fatoração                & 1317      & 135        & 9.8          \\
        Floresta Aleatória                   & 1002      & 145        & 22.3         \\
        Máquina de vetores de suporte        & 813       & 27         & 30.1         \\
        \textit{Naïve Bayes}                 & 106       & 3          & 35.3         \\
        \textit{Gradient boosting} em árvore & 2082      & 127        & 77.1         \\
        Perceptron multicamadas              & 1759      & 18         & 97.7         \\
        \hline
    \end{tabular}
    \caption{Tempo de treinamento dos modelos}
    \label{tab:modelos-tempo}
\end{table}

Modelos mais simples levam menos tempo para ser treinados pois exigem menos recursos computacionais e são preferidos quando a eficiências entre diferentes modelos forem parecidas.
Como vemos na tabela~\ref{tab:modelos-tempo} os modelos de regressão logística e máquinas de fatoração foram mais eficientes que os demais modelos, enquanto \textit{Gradient boosting} em árvore e Perceptron multicamadas tiveram um custo maior que os demais.

\subsection{\textit{Accuracy}}
\label{subsec:accuracy}

\begin{table}[h]
    \centering
    \begin{tabular}{|l|r|c|}
        \hline
        Modelo                               & Tempo/modelo & \textit{Accuracy}  \\
        \hline
        \textit{Gradient boosting} em árvore & 77.1         & 0.6788676814045260 \\
        Floresta Aleatória                   & 22.3         & 0.6787368124345655 \\
        Máquina de vetores de suporte        & 30.1         & 0.6784046065877426 \\
        Perceptron multicamadas              & 97.7         & 0.6780019328340179 \\
        Máquinas de fatoração                & 9.8          & 0.6767838447290005 \\
        Regressão Logística                  & 8.4          & 0.6764516388821776 \\
        \textit{Naïve Bayes}                 & 35.3         & 0.6735221873238302 \\
        \hline
    \end{tabular}
    \caption{\textit{Accuracy} dos modelos}
    \label{tab:modelos-accuracy}
\end{table}

Na tabela~\ref{tab:modelos-accuracy} todos os modelos tem uma \textit{Accuracy} bem próximas.
A diferença entre o melhor e o pior modelo é de 0.00534549408, ou 0.534549408\%.
Vamos analisar as outras métricas para diferenciar melhor os modelos.

\subsection{\textit{Precision}}
\label{subsec:precision}

\begin{table}[h]
    \centering
    \begin{tabular}{|l|r|c|}
        \hline
        Modelo                               & Tempo/modelo & \textit{Precision} \\
        \hline
        Perceptron multicamadas              & 97.7         & 0.6878171221735055 \\
        Máquina de vetores de suporte        & 30.1         & 0.6876848138152023 \\
        \textit{Gradient boosting} em árvore & 77.1         & 0.6872194787526914 \\
        Floresta Aleatória                   & 22.3         & 0.6863533577950337 \\
        Máquinas de fatoração                & 9.8          & 0.6833203199189651 \\
        Regressão Logística                  & 8.4          & 0.6801860252670480 \\
        \textit{Naïve Bayes}                 & 35.3         & 0.6735045664401927 \\
        \hline
    \end{tabular}
    \caption{\textit{Precision} dos modelos}
    \label{tab:modelos-precision}
\end{table}

Ao analisar a \textit{Precision} na tabela~\ref{tab:modelos-precision}, vê-se que a diferença entre o melhor e o pior modelo é de 0.01431255573 ou 1.431255573\%.
Ainda uma diferença muito pequena.

\subsection{\textit{Recall}}
\label{subsec:recall}

\begin{table}[h]
    \centering
    \begin{tabular}{|l|r|c|}
        \hline
        Modelo                               & Tempo/modelo & \textit{Recall}    \\
        \hline
        \textit{Naïve Bayes}                 & 35.3         & 0.9993717747363697 \\
        Regressão Logística                  & 8.4          & 0.9800762844962979 \\
        Máquinas de fatoração                & 9.8          & 0.9686784832847206 \\
        Floresta Aleatória                   & 22.3         & 0.9624859771146511 \\
        \textit{Gradient boosting} em árvore & 77.1         & 0.9595991324508264 \\
        Máquina de vetores de suporte        & 30.1         & 0.9566075835763966 \\
        Perceptron multicamadas              & 97.7         & 0.9550220626729489 \\
        \hline
    \end{tabular}
    \caption{\textit{Recall} dos modelos}
    \label{tab:modelos-recall}
\end{table}

Analisando o \textit{Recall} na tabela~\ref{tab:modelos-recall} vemos que todos os modelos tem um índice muito alto, maior que 95\%.
O modelo \textit{Naïve Bayes} tem um valor muito próximo de 1, ou seja, ele consegue detectar quase todos os casos que são positivos.

\subsection{\textit{F1 Score}}
\label{subsec:f1-score}

\begin{table}[h]
    \centering
    \begin{tabular}{|l|r|c|}
        \hline
        Modelo                               & Tempo/modelo & \textit{F1 Score}  \\
        \hline
        \textit{Naïve Bayes}                 & 35.3         & 0.8046995911042593 \\
        Regressão Logística                  & 8.4          & 0.8030468299976713 \\
        Máquinas de fatoração                & 9.8          & 0.8013537174640690 \\
        Floresta Aleatória                   & 22.3         & 0.8012975773160572 \\
        \textit{Gradient boosting} em árvore & 77.1         & 0.8008838509937083 \\
        Máquina de vetores de suporte        & 30.1         & 0.8001551415666796 \\
        Perceptron multicamadas              & 97.7         & 0.7996893826480128 \\
        \hline
    \end{tabular}
    \caption{\textit{F1 Score} dos modelos}
    \label{tab:modelos-f1}
\end{table}

E analisando o \textit{F1 Score}, que é uma métrica média entre \textit{Precision} e \textit{Recall}, o Naïve Bayes continua sendo o melhor modelo.
O \textit{Recall} muito alto em todos os modelos, acaba influenciando o \textit{F1 Score} e a classificação dos modelos por \textit{Recall} tem a mesma ordem que a classificação por F1 Score.

\section{Análise da Regressão Logística}
\label{sec:analise-da-regressao-logistica}

Dentre os modelos descritos, o modelo de regressão logística é um modelo que gera coeficientes com uma relação de mais fácil interpretabilidade.
Ao executar o código presente no repositório do GitHub deste trabalho~\cite{githubtcc}, foi obtido um modelo com os coeficientes presentes nas tabelas~\ref{tab:int-lr} e~\ref{tab:coef-lr}.

\begin{table}[h]
    \centering
    \begin{tabular}{|c|c|c|}
        \hline
        Coeficiente & Valor               & Probabilidade (\%) \\ \hline
        Intercepto  & -0.9615349252132588 & 27.66              \\ \hline
    \end{tabular}
    \caption{Intercepto do modelo de regressão logística}
    \label{tab:int-lr}
\end{table}

\begin{table}[h]
    \texttt{\scriptsize
    \centering
        \begin{tabular}{|l|l|r|r|}
            \hline
            Coeficiente & Valor & \makecell{Probabilidade \\relativa (\%)}            & \makecell{Probabilidade   \\real (\%)}                        \\ \hline
            Doença Renal Crônica & 0.4947658046512724  & 164.01 & 45.36 \\
            Puérpera         & 0.4554332099218894  & 157.69 & 43.61 \\
            Imunodeficiência/Imunodepressão               & 0.44086903459070775   & 155.41 & 42.98 \\
            Doença Hepática Crônica     & 0.43111209550372803   & 153.90 & 42.56 \\
            Diabetes mellitus                       & 0.3855162684195613  & 147.04 & 40.67 \\
            Doença Hematológica Crônica                            & 0.3315345906093188  & 139.31 & 38.53 \\
            Obesidade      & 0.30353342237041214  & 135.46 & 37.46 \\
            Asma     & 0.20758009401090677  & 123.07 & 34.04 \\
            Doença Neurológica Crônica                & 0.19574036293390837 & 121.62  & 33.64 \\
            Outra Pneumatopatia Crônica   & 0.12989092233370036  & 113.87  & 31.49 \\
            Síndrome de Down   & -0.14507414792834297  & 86.50  & 23.92 \\
            Doença Cardiovascular Crônica   & -0.3530147635129998  & 70.26  & 19.43 \\ \hline
        \end{tabular}
        \caption{Coeficientes do modelo de regressão logística}
        \label{tab:coef-lr}
    }
\end{table}

De acordo com o intercepto na tabela~\ref{tab:int-lr} e aplicando a equação~\ref{eq:int-rl}, aproximadamente 27,66\% das pessoas internadas com COVID-19 sem nenhuma comorbidade não sobreviverão.
Na tabela~\ref{tab:coef-lr}, ordenada pelo valor dos coeficientes de maneira decrescente, temos a probabilidade relativa, calculada com a equação~\ref{eq:prob-rl} e a probabilidade real, calculada multiplicando a probabilidade relativa pela probabilidade base, calculada pelo intercepto.
Vemos que quase todas as comorbidades aumentam o risco de morte exceto duas: Síndrome de Down e Doença Cardiovascular Crônica.

Algo que chama a atenção é o fato de pessoas com doenças cardiovasculares crônicas terem um menor risco de morte se comparado com quem não tem nenhuma comorbidade.
Isto pode ser devido ao fato de pessoas com doenças cardiovasculares crônicas procurarem ajuda médica mais cedo devido às campanhas médicas presentes que indicam maior probabilidade de desenvolver a forma grave da doença.
Como este trabalho utilizou dados de pessoas já hospitalizadas, não está sendo levado em conta as pessoas com comorbidades que tiveram sintomas leves ou assintomáticas.

