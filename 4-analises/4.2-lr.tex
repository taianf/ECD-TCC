\section{Análise da Regressão Logística}
\label{sec:analise-da-regressao-logistica}

Dentre os modelos descritos, o modelo de regressão logística é um modelo que gera coeficientes com uma relação de mais fácil interpretabilidade.
Ao executar o código presente no repositório do GitHub deste trabalho~\cite{githubtcc}, foi obtido um modelo com os coeficientes presentes nas tabelas~\ref{tab:int-lr} e~\ref{tab:coef-lr}.

\begin{table}[h]
    \centering
    \begin{tabular}{|c|c|c|}
        \hline
        Coeficiente & Valor               & Probabilidade (\%) \\ \hline
        Intercepto  & -0.9615349252132588 & 27.66              \\ \hline
    \end{tabular}
    \caption{Intercepto do modelo de regressão logística}
    \label{tab:int-lr}
\end{table}

\begin{table}[h]
    \texttt{\scriptsize
    \centering
        \begin{tabular}{|l|l|r|r|}
            \hline
            Coeficiente & Valor & \makecell{Probabilidade \\relativa (\%)}            & \makecell{Probabilidade   \\real (\%)}                        \\ \hline
            Doença Renal Crônica & 0.4947658046512724  & 164.01 & 45.36 \\
            Puérpera         & 0.4554332099218894  & 157.69 & 43.61 \\
            Imunodeficiência/Imunodepressão               & 0.44086903459070775   & 155.41 & 42.98 \\
            Doença Hepática Crônica     & 0.43111209550372803   & 153.90 & 42.56 \\
            Diabetes mellitus                       & 0.3855162684195613  & 147.04 & 40.67 \\
            Doença Hematológica Crônica                            & 0.3315345906093188  & 139.31 & 38.53 \\
            Obesidade      & 0.30353342237041214  & 135.46 & 37.46 \\
            Asma     & 0.20758009401090677  & 123.07 & 34.04 \\
            Doença Neurológica Crônica                & 0.19574036293390837 & 121.62  & 33.64 \\
            Outra Pneumatopatia Crônica   & 0.12989092233370036  & 113.87  & 31.49 \\
            Síndrome de Down   & -0.14507414792834297  & 86.50  & 23.92 \\
            Doença Cardiovascular Crônica   & -0.3530147635129998  & 70.26  & 19.43 \\ \hline
        \end{tabular}
        \caption{Coeficientes do modelo de regressão logística}
        \label{tab:coef-lr}
    }
\end{table}

De acordo com o intercepto na tabela~\ref{tab:int-lr} e aplicando a equação~\ref{eq:int-rl}, aproximadamente 27,66\% das pessoas internadas com COVID-19 sem nenhuma comorbidade não sobreviverão.
Na tabela~\ref{tab:coef-lr}, ordenada pelo valor dos coeficientes de maneira decrescente, temos a probabilidade relativa, calculada com a equação~\ref{eq:prob-rl} e a probabilidade real, calculada multiplicando a probabilidade relativa pela probabilidade base, calculada pelo intercepto.
Vemos que quase todas as comorbidades aumentam o risco de morte exceto duas: Síndrome de Down e Doença Cardiovascular Crônica.

Algo que chama a atenção é o fato de pessoas com doenças cardiovasculares crônicas terem um menor risco de morte se comparado com quem não tem nenhuma comorbidade.
Isto pode ser devido ao fato de pessoas com doenças cardiovasculares crônicas procurarem ajuda médica mais cedo devido às campanhas médicas presentes que indicam maior probabilidade de desenvolver a forma grave da doença.
Como este trabalho utilizou dados de pessoas já hospitalizadas, não está sendo levado em conta as pessoas com comorbidades que tiveram sintomas leves ou assintomáticas.
